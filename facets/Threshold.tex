\documentclass[11pt]{amsart} 
\usepackage{amssymb,amsmath,latexsym,enumerate,graphicx,bbm,mathptmx,microtype,cite}
\allowdisplaybreaks

\hoffset=0in 
\voffset=0in
\oddsidemargin=0in
\evensidemargin=0in
\topmargin=0.3in 
\headsep=0.15in 
\headheight=8pt
\textwidth=6.5in
\textheight=8.5in

\usepackage{float}

\usepackage{amsthm, pict2e, hyperref, mathtools }

\usepackage{xcolor}
\newlength\mylen
\usepackage{array} % for "\newcolumntype" macro
\newcolumntype{C}{>>{\hfil$}p{\mylen}<{$\hfil}}
%\makeatletter
\usepackage{tikz}
\usetikzlibrary{graphs,graphs.standard}
\usepackage{tikz-cd}
\usepackage{caption}
\usepackage{ulem}
\usepackage{color}
\usepackage{scrlayer-scrpage}
\usepackage{siunitx}
\usepackage{lastpage}
\usepackage{comment}
\usetikzlibrary{arrows,intersections,patterns}
\usetikzlibrary{decorations.markings}
%%%%%%%%%%%%%%%%%%%%%%%%%%%%%% Textclass specific LaTeX commands.

\newtheorem{theorem}{Theorem} % [section] \numberwithin{equation}{section}
\newtheorem{cor}[theorem]{Corollary}
\newtheorem{prop}[theorem]{Proposition}
\newtheorem{lem}[theorem]{Lemma}
\newtheorem{conjecture}{Conjecture}
\newtheorem*{exam}{Example}
\newenvironment{example}{\begin{exam}\rm}{\end{exam}}
\newtheorem*{rem}{Remarks}
\newenvironment{remark}{\begin{rem}\rm}{\end{rem}}

\newcommand\lcm{\operatorname{lcm}} 
\newcommand\vol{\operatorname{vol}} 
\newcommand\conv{\operatorname{conv}} 
\newcommand\vspan{\operatorname{span}} 
\newcommand\const{\operatorname{const}} 
\newcommand\cone{\operatorname{cone}} 
\newcommand\link{\operatorname{link}} 
\newcommand\height{\operatorname{height}} 
\newcommand\ehr{\operatorname{ehr}} 
\newcommand\Ehr{\operatorname{Ehr}} 
\newcommand\SL{\operatorname{SL}} 

\newcommand\ZZ{\mathbb{Z}}
\newcommand\QQ{\mathbb{Q}}
\newcommand\RR{\mathbb{R}}
\newcommand\CC{\mathbb{C}}

\newcommand\cC{\mathcal{C}}
\newcommand\cF{\mathcal{F}}
\newcommand\cG{\mathcal{G}}
\newcommand\cH{\mathcal{H}}
\newcommand\cK{\mathcal{K}}
\newcommand\cL{\mathcal{L}}
\newcommand\cM{\mathcal{M}}
\newcommand\cP{\mathcal{P}}
\newcommand\cR{\mathcal{R}}
\newcommand\cS{\mathcal{S}}
\newcommand\cT{\mathcal{T}}



\newcommand\bA{\mathbf{A}}
\newcommand\bB{\mathbf{B}}
\newcommand\ba{\mathbf{a}}
\newcommand\bb{\mathbf{b}}
\newcommand\bc{\mathbf{c}}
\newcommand\be{\mathbf{e}}
\newcommand\bm{\mathbf{m}}
\newcommand\bU{\mathbf{U}}
\newcommand\bu{\mathbf{u}}
\newcommand\bv{\mathbf{v}}
\newcommand\bw{\mathbf{w}}
\newcommand\bx{\mathbf{x}}
\newcommand\by{\mathbf{y}}
\newcommand\bzero{\mathbf{0}}

\theoremstyle{definition}
\newtheorem{definition}{Definition}[section]
\theoremstyle{definition}

%%%%%%%%%%%%%%%%%%%%%%%%%%%%%% User specified LaTeX commands.

\usetikzlibrary{arrows.meta}
\usepackage{tikz-cd}
\usepackage{graphicx}
\usepackage{caption}

\newcommand{\COMMENT}[2][.2\linewidth]{%
  \leavevmode\hfill\makebox[#1][l]{//~#2}}
\usepackage[ruled, linesnumbered, noend]{algorithm2e}

%%%%%%%%%%%%%%%%%%%%%%%%%%%%%%%%%% math operators

\DeclareMathOperator{\im}{im}
\DeclareMathOperator{\rank}{rank}
\DeclareMathOperator{\indeg}{indeg}
\DeclareMathOperator{\outdeg}{outdeg}

\makeatletter % to repeat theorem numbers
\newtheorem*{rep@theorem}{\rep@title}\newcommand{\newreptheorem}[2]{%
\newenvironment{rep#1}[1]{%
\def\rep@title{\bf #2 \ref{##1}}%
\begin{rep@theorem}}%
{\end{rep@theorem}}}
\makeatother
\newreptheorem{theorem}

\renewcommand\emptyset{\varnothing}
\renewcommand\th{^{\text{th}}}

\newcommand\fl[1]{\left\lfloor {#1} \right\rfloor} 
\newcommand\fr[1]{\left\{ {#1} \right\}} 
\newcommand\commentout[1]{}
\newcommand\Def[1]{{\bf #1}}
\newcommand\set[2]{{#1} : \, {#2}}



\title{The number of hyperplanes spanned by linear independent zero-one vectors}



\begin{document}

\maketitle

Let $c \in \mathbb{R}^n$ and $T \in \mathbb{R}$. 
A \Def{Boolean threshold function} $f \colon \{0,1 \}^n \rightarrow \{0,1 \}$ is a function defined by

\begin{equation*}
  f(x)=\begin{cases}
    1, & \text{if $c^Tx \geq T$}\\
    0, & \text{if $c^Tx < T$}.
  \end{cases}
\end{equation*}

The hyperplane $\{ x \in \mathbb{R}^n : c^Tx = T \} $ divides the cube $[0,1]^n$ into two (possibly empty) polyhedra $S_1 \coloneqq \{ x \in [0,1]^n : c^Tx \geq T \}$, and $S_2 \coloneqq \{ x \in [0,1]^n : c^Tx < T \} $. 
Thus, $f$ applied to any vertex of $S_1$ is $1$ and applied to any vertex of $S_2$ is $0$. 
We say that $\{ x \in \mathbb{R}^n : c^Tx = T \} $ is the \Def{hyperplane induced by $f$}.

Consider the LP 
\begin{align*}
    \text{for all  } x \in S_1: c^Tx \geq 0 \text{ and} \\
    \text{for all  } x \in S_2: c^Tx \leq -1 \\
\end{align*}

is a basis solution $v_1,...,v_i,v_{i+1},...,v_n$ where $v_1,...,v_i \in S_1 $ and $v_{i+1},..., v_n \in S_2$ are linearly independent and, as basis solution they satisfy $c^Tv_j = 0$ for all $j \in \{1,...,i\}$ and $c^Tv_j = -1$ for all $j \in \{i+1,...,n\}$.

Consider the "lifted" hyperplane $\{ (x^T, x_{n+1})^T \in \mathbb{R}^{n+1} : c^Tx - Tx_{n+1} \geq 0 \} $ in $\mathbb{R}^{n+1}$ defining the polyhedra $S_1^{new} \subseteq \mathbb{R}^{n+1}$ and $S_2^{new} \subseteq \mathbb{R}^{n+1}$ as we defined $S_1$ and $S_2$ before. It follows that $S_1 = \{ x \in \mathbb{R}^n : (x^T,1)^T \in S_1^{new} \}  $ and $S_2 = \{ x \in \mathbb{R}^n : (x^T,0)^T \in S_2^{new} \}  $. 

Consider the lifted basis solution 
\[ \left( \begin{array}{cc}
v_1\\
0 
\end{array} \right),\dots,
%
\left( \begin{array}{cc}
v_i\\
0 
\end{array} \right),\dots,
\left( \begin{array}{cc}
v_{i+1}\\
1
\end{array} \right),\dots,
\left( \begin{array}{cc}
v_n\\
1
\end{array} \right)
\]

This solution defines as well a hyperplane through $0$ in $\mathbb{R}^{n+1}$ as a threshold function.  Thus, the number of hyperplanes spanned by $(n-1)$ linear independent $0/1$ vectors is bigger or equal the number of Boolean threshold functions. A lower bound on the number of Boolean threshold functions is given by the following Proposition. 

\begin{prop}
The number of Boolean threshold functions is bigger or equal $2^{\frac{1}{2}n^2}$. 
\end{prop}

The proof follows by the following lemmas.

\begin{lem} \label{lem1}
Let $f,g \colon \{0,1 \}^n \rightarrow \{0,1 \}$ and $f',g' \colon \{0,1 \}^n \rightarrow \{0,1 \}$ be Boolean threshold functions. Define two Boolean threshold functions $h,h' \colon \{0,1 \}^{n+1} \rightarrow \{0,1 \}$ as 
\begin{align*}
    h(x,x_{n+1}) = f(x)x_{n+1} + g(x)(1-x_{n+1}) \\
    h'(x,x_{n+1}) = f'(x)x_{n+1} + g'(x)(1-x_{n+1}).
\end{align*}
Then if $f\neq f' $ or $g \neq g'$ it follows that $h \neq h'$. 
\end{lem}

\begin{lem} \label{lem2}
Let $c \in \mathbb{R}^n$ and $T_f,T_g \in \mathbb{R}$ be given. 
Let $f,g \colon \{0,1 \}^n \rightarrow \{0,1 \}$ be Boolean threshold functions defined as
\begin{equation*}
  f(x)=\begin{cases}
    1, & \text{if $c^Tx \geq T_f$}\\
    0, & \text{if $c^Tx < T_f$}
  \end{cases}
\end{equation*} and 

\begin{equation*}
  g(x)=\begin{cases}
    1, & \text{if $c^Tx \geq T_g$}\\
    0, & \text{if $c^Tx < T_g$}.
  \end{cases}
\end{equation*}
Then $h$, as defined, in Lemma \ref{lem1} is a threshold function.

\end{lem}

\begin{proof}
If $x_{n+1} = 1$, then $h(x,1) = 1$ if and only if $c^Tx \geq T_fx_{n+1}$. If $x_{n+1} = 0$, then $h(x,0) = 1$ if and only if $c^Tx \geq T_gx_{n+1}$. Thus, $h(x,x_{n+1})=1$ if and only if $c^Tx+(T_g-T_f)x_{n+1} \geq T_g$. 
\end{proof}

Now let $f \colon \{0,1 \}^n \rightarrow \{0,1 \}$ be a given Boolean threshold function with $c \in \mathbb{R}^n$ and we assume w.l.o.g. that $c^Tx \neq c^Ty$ for all $x,y \in \{0,1 \}^n$ with $x \neq y$. 
Define a series of Boolean threshold functions $g_i \colon \{0,1 \}^n \rightarrow \{0,1 \}$ such that the hyperplane induced by $g_i$ cuts off $i \in \{0,..., 2^n \}$ many integer points of $[0,1]^n$. So, there are $2^n+1$ different $g_i$. By Lemma \ref{lem2}, the functions $h^{f,i}$ defined by $h^{f,i}(x,x_{n+1}) = f(x)x_{n+1} + g_i(x)(1-x_{n+1})$ are Boolean threshold functions. By Lemma \ref{lem1}, $h^{f,i} \neq h^{f',i'}$ if and only if $f \neq f'$ or $i \neq i'$. Thus, for the number $T(n+1)$ of Boolean threshold functions we have $T(n+1) \leq (2^n+1)T(n) \leq... \leq 2^n \cdot 2^{n-1} \cdots 2^0 = 2^{\frac{1}{2}n^2}$, by the geometric series.




















\end{document}
