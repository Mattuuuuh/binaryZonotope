\documentclass[12pt]{report}
\usepackage{amsmath,amssymb,amsfonts,amsthm} %Mathesymbole
\usepackage{enumerate}
\usepackage{fancybox}
\usepackage{tikz}
\usetikzlibrary{calc} 
\usetikzlibrary{arrows}
\usepackage{algorithm,algorithmic}

\newtheorem*{remark}{Remark}
\newtheorem*{hypothesis}{Hypothesis}
\newtheorem*{lemma}{Lemma}
\newtheorem*{proposition}{Proposition}
\newtheorem*{definition}{Definition}
\newtheorem*{notation}{Notation}
\newtheorem*{corollary}{Corollary}
\newtheorem*{example}{Example}
\newtheorem*{theorem}{Theorem}

% Schwartz
\renewcommand{\S}{\mathcal{S}} % \S est le signe paragraphe normalement

% corps
\newcommand{\C}{\mathbb{C}}
\newcommand{\R}{\mathbb{R}}
\newcommand{\Rnn}{\mathbb{R}^{2n}}
\newcommand{\Z}{\mathbb{Z}}
\newcommand{\N}{\mathbb{N}}

% domain
\newcommand{\D}{\mathcal{D}}

% volume, rank
\newcommand{\Vol}{\emph{Vol}}
\newcommand{\rank}{\emph{rank}}

% order notations
\renewcommand{\O}{\mathcal{O}}

% japanese bracket
\newcommand{\japb}[1]{\langle #1 \rangle}

% arrows over partial derivatives
\newcommand{\lp}{\overleftarrow{\partial}}
\newcommand{\rp}{\overrightarrow{\partial}}

% quantization
\newcommand{\h}{\hbar}
\newcommand{\Opht}{\textrm{Op}_{\h}^{t}}
\newcommand{\Op}[2][\hbar]{\textrm{Op}_{#1}^{#2}}

% omega functions
\newcommand{\omegap}[2][\rho_0]{\omega(\partial_{#1},\partial_{#2})}
\newcommand{\omegar}[2][\rho_0]{\omega(#1,#2)}

% fourier constants
\newcommand{\F}{\mathcal{F}}
\renewcommand{\c}{\frac{1}{(2\pi\h)^{n/2}}}
\newcommand{\cc}{\frac{1}{(2\pi\h)^n}}
\renewcommand{\d}{\frac{1}{(2\pi)^{n}}}
\newcommand{\dd}{\frac{1}{(2\pi)^{2n}}}

% sup inf sum
\newcommand{\ssup}{\sup\limits}
\newcommand{\iinf}{\inf\limits}
\newcommand{\ssum}{\sum\limits}

% norm
\newcommand{\norm}[1]{\left\Vert #1 \right\Vert}

\newcommand{\mZ}{\mathcal{Z}}

\begin{document}

\begin{definition}
We set $\mZ_m$ the zonotope generated by all $v \in \{0, 1\}^m$.
\end{definition}

We aim to prove the following theorem.
\begin{theorem}
The number of facets $F(m)$ of $\mZ_m$ is at least $2^{\frac12 m log(m) + \mathcal{O}(m loglog(m))}$.
\end{theorem}

\begin{lemma}
The following equivalences hold.

\begin{center}
\text{c is a facet definining vector} \\
$\iff$ \\
$c \in \R^m, dim(c^\perp \cap \{0, 1\}^m) = m-1$ \\ 
$\iff$ \\
$\exists \{v_1, \dots, v_{m-1}\} \text{ linearly independent vectors of } \{0, 1\}^m \text{ perpendicular to c.}$
\end{center}

There further is a bijection between each facet and all positive multiples of some facet-defining vector $c$.
\end{lemma}
We may suppose $c \in \Z^m$ since there exist an integer solution to $Ac = 0$ as soon as $A$ is rational. To get a unique $c$, we may suppose the coordinates of $c$ are coprime.

\begin{notation}
$c^\perp := \{ x \in \R^m | c^T x=0 \} \cap \{0, 1\}^m$, \\ $Im(c) := \{ c^T v | v \in \{0, 1\}^m\}$   \\ We use "facet" colloquially for a facet-defining vector $c$. \\ We note $\begin{pmatrix} 0 \\ c^\perp \end{pmatrix} := \{ \begin{pmatrix} 0 \\ v \end{pmatrix} | v \in c^\perp \}$.
\end{notation}

\begin{proposition}
Each facet $c \in \Z^m$ of $\mZ_m$ generates $| Im (c) |$ unique facets of $\mZ_{m+1}$ in $\Z^{m+1}$.
\end{proposition}
\begin{proof}
Let $Im(c) = \{ \alpha_1, \dots, \alpha_i \}$, with $\alpha_1=0$. And define $d_k := \begin{pmatrix} - \alpha_k \\ c \end{pmatrix}$.

Then each $d_k$ is a facet of $\mZ_{m+1}$. Indeed, $d_k \neq 0$ and $d_k^\perp$ includes both $\begin{pmatrix} 0 \\ c^\perp \end{pmatrix}$ and some vector of the form $\begin{pmatrix} 1 \\ v \end{pmatrix} \in \{0, 1\}^{m+1}$. It hence includes $m$ linearly independant vectors and generates a facet.

We must show the facets are unique and that there is no overlap from two different facets of $\mZ_m$. We cannot have $d_k^\perp = d_l^\perp$ for some $l \neq k$, since the added vector $\begin{pmatrix} 1 \\ v \end{pmatrix} \in \{0, 1\}^{m+1}$ is not perpendicular to $d_l$.

For two different starting facets $c, c' \in \Z^m$, the generated facets of $\mZ_{m+1}$ are also different since the vectors perpendicular to them with first coordinate 0 are exactly $\begin{pmatrix} 0 \\ c^\perp \end{pmatrix}$ and $\begin{pmatrix} 0 \\ (c')^\perp \end{pmatrix}$ respectively. 
\end{proof}

\begin{definition}
$F(m) :=  f_{m-1}(\mZ_m) = $ number of facets of $\mZ_m$.

$\alpha_l := \{ c$ facets of $\mZ_m : |Im(c)| = l \}$.
\end{definition}

Note that the size of the image is invariant under positive multiplication. Also, $\sum_{l \geq 2} \alpha_l = F(m)$, and the sum is finite. It starts at $2$ since $dim(c^\perp) = m-1$.

\begin{corollary}
\[
F(m+1) \geq \sum_{l \geq 2} l \alpha_l.
\]

And hence $F(m+1) \geq 2 F(m)$.
\end{corollary}

\begin{example}
Let's compute the first coefficient $\alpha_2$. 

Take a $c \in \Z^m$ such that $Im(c) = \{ 0, \alpha \}$, $\alpha \in \Z$, and decompose it as
\[
c = (0, 0, \dots, 0, \tilde{c}),
\]
with $\tilde{c}_k \neq 0 \forall k$ and up to a choice of placement of the nonzero coefficients.

Note that $c$ is a facet of $\mZ_m$ if and only if $\tilde{c} \in \Z^d$ is a facet of $\mZ_{d}$. Furthermore, note that $Im(c) = Im(\tilde{c}) = \{0, \alpha\}$. However, since $\tilde{c}_k$ is in the image for each k, and it is nonzero, we must have $\tilde{c} = (\alpha, \dots, \alpha)$. In this case, $Im(\tilde{c}) = \{0, \alpha, 2\alpha, \dots d \alpha\}$, which implies directly that $d=1$. Up to normalization, the only two possible $\tilde{c}$ are in fact $1$ and $-1$.

This implies in turn that only $2m$ vectors $c$ exist with the supposed properties, and $\alpha_2 = 2m$.
\end{example}

\begin{lemma}
\[
\alpha_l \leq l m^{l^2} F(l^2).
\]
\end{lemma}
\begin{proof}
We split $c \in \Z^m$ as
\[
c = (0, 0, \dots, 0, \tilde{c}),
\]
with $\tilde{c}_k \neq 0 \forall k$ and up to a choice of placement of the nonzero coefficients. We aim now to bound the dimension $d$ that $\tilde{c} \in \Z^d$ defines a facet in.

Since $|Im(c)| = l$, each $\alpha \in Im(c)$ may appear in $\tilde{c}$ at most $l$ times. This gives directly that $d \leq l^2$.

Hence 
\[
\alpha_l \leq \sum_{i=1}^{l^2} {m \choose i} F(i) \leq l^2 m^{l^2} F(l^2).
\]
\end{proof}

\begin{proposition}
\[
F(m+1) \geq \frac14 \sqrt{\frac{m}{log(m)}} F(m),
\]
for all $m$ sufficiently large.
\end{proposition}
\begin{proof}
Set the threshold $T := \frac12 \sqrt{\frac{m}{log(m)}}$. The corollary above gives
\begin{align*}
F(m+1) \geq \sum_{l \geq 2} l \alpha_l &\geq \sum_{l=2}^{T-1} l \alpha_l + T \sum_{l\geq T} \alpha_l, \\
							 &\geq \sum_{l=2}^{T-1} l \alpha_l + T \left[ F(m) - \sum_{l=2}^{T-1} \alpha_l \right], \\
							 &= T F(m) + \sum_{l=2}^{T-1} (l-T) \alpha_l.
\end{align*}
By the proposition, $\alpha_l \leq l m^{l^2} F(l^2)$, and therefore
\[
(l-T) \alpha_l \geq -T l^2 m^{l^2} F(l^2).
\]
Furthermore, since $F(m+1) \geq 2F(m)$, we get
\[
(l-T) \alpha_l \geq -T l^2 m^{l^2} 2^{l^2 - m} F(m).
\]
Since $l\leq T=  \frac12 \sqrt{\frac{m}{log(m)}}$, we evaluate the expression to
\[
(l-T) \alpha_l \geq - m^{\frac32} 2^{\frac14 m} 2^{\frac14 m - m} F(m) \geq - 2^{-\frac14 m} F(m),
\]
for $m$ sufficiently large.

In conclusion, 
\begin{align*}
F(m+1) \geq \sum_{l \geq 2} l \alpha_l &\geq  T F(m) + \sum_{l=2}^{T-1} (l-T) \alpha_l, \\
							  &\geq T F(m) - 2^{-\frac14 m} F(m), \\
							  &\geq \frac12 T F(m) =  \frac14 \sqrt{\frac{m}{log(m)}} F(m),
\end{align*}
for $m$ again sufficiently large.
\end{proof}

\begin{corollary}
The number of facets of $\mZ_m$ is at least $2^{\frac12 m log(m) + \mathcal{O}(m loglog(m))}$.
\end{corollary}
\begin{proof}
Since $F(m+1) \geq \frac14 \sqrt{\frac{m}{log(m)}} F(m)$ for all $m > N$ for some constant $N$, 
\begin{align*}
F(m) &\geq 4^{-m} \sqrt{\frac{m!}{\prod_{k=2}^{m} log(k)}} \sqrt{\frac{\prod_{k=2}^{N} log(k) }{N!}} F(N), \\
	&= 2^{\frac12 m log(m) + \mathcal{O}(m loglog(m))},
\end{align*}
because $\prod_{k=2}^{m} log(k) \leq 2^{m loglog(m)}$ (tight).
\end{proof}


\end{document}